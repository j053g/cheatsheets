\section*{Week 6A: Financial Statement Analysis and Ratios}

\subsection*{Solvency and Liquidity Ratios}

$Debt/EquityRatio = \frac{Total Liabilibies}{Total Shareholder's Equity}$ \\

$Leverage Ratio = \frac{TotalAssets}{Total Shareholder's Equity}$ \\

$Leverage Ratio = \frac{A}{L}=\frac{E+L}{L} = 1+ Debt/EquityRatio$ 


$Current Ratio = \frac{Current Assets}{Current Liabilities}$ \\


$Working Capital = \\ Current Assets - Current Liabilities $ 



\subsection*{Solvency and Liquidity Ratios}

$Net Margin = \frac{Net Income}{Revenue}$ \\

$Gross Margin = \frac{Revenue - COGS}{Revenue}$ \\

$ROA = \frac{Net Income}{Total Assets}$ \\

$ROE = \frac{Net Income}{Shareholders' Equity}$ 

\subsection*{Operating Efficiency}

$Asset Turnover = \frac{Revenue}{Total Assets}$  \\

$A/R Turnover = \frac{Revenue}{Net Accounts Receivable}$  \\

$Inventory Turnover = \frac{COGS}{Inventory}$  \\

$Days Receivable = \frac{365}{A/R Turnover}$  

\subsection*{DuPont Decomposition}

$ROE = \frac{NI}{Equity}$ \\

$ROE = \frac{NI}{Equity} \frac{Assets}{Equity} = ROA \cdot Leverage$  \\

$ROE = \frac{NI}{Sales} \frac{Sales}{Assets}\frac{Assets}{Equity} \\
 ROE = ProfitMargin \cdot AssetTurnover \cdot Leverage $  \\
 

\section*{Week 6B: Income Taxes}

$AccountingIncome \neq Taxable Income$ \\
$Tax Expense \neq Cash Taxes $ \\


\subsection*{Deferred Tax Liability - DTL}

Deferred tax liabilities increase when a timing difference leads to: \\
 $PretaxIncome_{GAAP}>Taxable Income_{TaxCode}	$

Balance sheet equation for cash paid in tax vs tax liability and income tax expense:
\begin{tabular}{ |c||c|c| } 
	\hline
	  Assets = & Liab +  & S/E	 \\ 
	\hline
	  CashTax & DefTaxLiab  & InTaxExp	 \\ 
   	
	\hline
\end{tabular}

If a company has a net deferred tax liability on its balance sheet, cash taxes in the future will be higher than future tax expense.
 
\subsection*{ Deferred Tax Assets - DTA}
 
Deferred tax assets increase when a timing difference leads to: \\
 $PretaxIncome_{GAAP}<Taxable Income_{TaxCode}	$ \\
more tax cash early, less cash taxes later.  $DefTaxAsset$ is similar to prepaid expense:
 
\begin{tabular}{ |c||c|c| } 
	\hline
	Assets = & Liab +  & S/E	 \\ 
	\hline
	CashTax  DefTaxAsset &  & InTaxExp	 \\ 
	
	\hline
\end{tabular}   

\subsection*{ Tax Disclosures}

When the tax rate falls, the  DTA (or DTL) shrinks. When a company has net DTL, we can think of this shrinking DTL as a one-time tax benefit which will reduce tax expense.

The DTL will shrink by the ratio of the rates:  $\frac{tax_{new}}{tax_{old}}$ \\

$ DTL_{new} = DTL _{old} \frac{tax_{new}}{tax_{old}} $ \\

$ \Delta DTL = DTL_{new} – DTL _{old} = (1-\frac{tax_{new}}{tax_{old}})DTL _{old}$

 
\begin{tabular}{ |c||c|c| } 
	\hline
	Assets = & Liab +  & S/E	 \\ 
	\hline
	& $\Delta DTL$ &   $-\Delta DTL$	 \\ 
	
	\hline
\end{tabular}   

Similarly when a company has a net DTA: \\

$ \Delta DTA = DTA_{new} – DTA _{old} = (1-\frac{tax_{new}}{tax_{old}})DTA _{old}$


\begin{tabular}{ |c||c|c| } 
	\hline
	Assets = & Liab +  & S/E	 \\ 
	\hline
	 $\Delta DTA$ &  & $\Delta DTA$	 \\ 
	
	\hline
\end{tabular} 

\subsection*{ Effective Tax Rate}

$EffectiveTaxRate = \frac{TaxExpense}{GAAPpretaxIncome}$  \\
$ pretaxIncome = NetIncome + TaxExpense $ \\
$EffectiveTaxRate = \frac{TaxExpense}{TaxExpense+NetIncome}$  

\subsection*{DTAs and Valuation Allowance}

Deferred tax assets arise when future taxes payable will be less than future tax expense. DTAs are like “pre-paid” assets.
Firms reduce deferred tax assets by creating a valuation allowance, a contra-asset that is
similar to the allowance for doubtful accounts.\\


Example: In 2015, a firm has a \$30,000 deferred tax asset.
Suppose instead, at end of 2016, management expects that it will not have enough future income
to use the DTA:
\begin{tabular}{ |c|c||c|c| } 
	\hline
	Asset &-ContraAsset = & Liab +  & S/E	 \\ 
	\hline
	& $-30000$ &  & $-30000$	 \\ 	
	\hline
\end{tabular} 