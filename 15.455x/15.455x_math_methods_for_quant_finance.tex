\documentclass[10pt,landscape,a4paper]{article}
\usepackage[english]{babel}
\usepackage[utf8]{inputenc}
\usepackage[plain]{algorithm}
\usepackage[noend]{algpseudocode}
\usepackage{tikz}
\usepackage{pgfplots}
\usepackage{palatino}
\usepackage{multicol}
\usepackage{blkarray}

\usepackage{calc}
\usepackage{ifthen}
\usepackage[landscape]{geometry}
\usepackage{graphicx}
\usepackage{amsmath, amssymb, amsthm}
\DeclareMathOperator*{\argmin}{argmin}
\DeclareMathOperator*{\argmax}{argmax}

\usepackage{physics}
\usepackage{latexsym, marvosym}
\usepackage{pifont}
\usepackage{lscape}
\usepackage{dsfont}
\usepackage{graphicx}
\usepackage{array}
\usepackage{booktabs}
\usepackage[bottom]{footmisc}
\usepackage{tikz}
\usetikzlibrary{shapes}
\usepackage{pdfpages}
\usepackage{wrapfig}
\usepackage{enumitem}
\setlist[description]{leftmargin=0pt}
\usepackage{xfrac}
\usepackage[pdftex,
            pdfauthor={Jose},
            pdftitle={Mathematical Methods for Quantitative Finance},
            pdfsubject={Cheat sheet for MITx 15.455x Mathematical Methods for Quantitative Finance},
            pdfkeywords={finance} {cheatsheet} {pdf} {cheat} {sheet} {formulas} {equations}
            ]{hyperref}
\usepackage[
            open,
            openlevel=2
            ]{bookmark}
\usepackage{relsize}
\usepackage{rotating}

% for monospace font https://tex.stackexchange.com/questions/50810/good-monospace-font-for-code-in-latex
\usepackage{inconsolata}


 \newcommand\independent{\protect\mathpalette{\protect\independenT}{\perp}}
    \def\independenT#1#2{\mathrel{\setbox0\hbox{$#1#2$}%
    \copy0\kern-\wd0\mkern4mu\box0}} 
            
\newcommand{\noin}{\noindent}    
\newcommand{\logit}{\textrm{logit}} 
%\newcommand{\var}{\textrm{Var}}
\newcommand{\cov}{\textrm{Cov}} 
\newcommand{\corr}{\textrm{Corr}} 
\newcommand{\N}{\mathcal{N}}
\newcommand{\Bern}{\textrm{Bern}}
\newcommand{\Bin}{\textrm{Bin}}
\newcommand{\Beta}{\textrm{Beta}}
\newcommand{\Gam}{\textrm{Gamma}}
\newcommand{\Expo}{\textrm{Expo}}
\newcommand{\Pois}{\textrm{Pois}}
\newcommand{\Unif}{\textrm{Unif}}
\newcommand{\Geom}{\textrm{Geom}}
\newcommand{\NBin}{\textrm{NBin}}
\newcommand{\Hypergeometric}{\textrm{HGeom}}
\newcommand{\HGeom}{\textrm{HGeom}}
\newcommand{\Mult}{\textrm{Mult}}

\geometry{top=.4in,left=.2in,right=.2in,bottom=.4in}

\pagestyle{empty}
\makeatletter
\renewcommand{\section}{\@startsection{section}{1}{0mm}%
                                {-1ex plus -.5ex minus -.2ex}%
                                {0.5ex plus .2ex}%x
                                {\normalfont\large\bfseries}}
\renewcommand{\subsection}{\@startsection{subsection}{2}{0mm}%
                                {-1explus -.5ex minus -.2ex}%
                                {0.5ex plus .2ex}%
                                {\normalfont\normalsize\bfseries}}
\renewcommand{\subsubsection}{\@startsection{subsubsection}{3}{0mm}%
                                {-1ex plus -.5ex minus -.2ex}%
                                {1ex plus .2ex}%
                                {\normalfont\small\bfseries}}
\makeatother

\setcounter{secnumdepth}{0}

\setlength{\parindent}{0pt}
\setlength{\parskip}{0pt plus 0ex}

% -----------------------------------------------------------------------

\usepackage{titlesec}

\titleformat{\section}
{\color{blue}\normalfont\normalsize\bfseries}
{\color{blue}\thesection}{0em}{}
\titleformat{\subsection}
{\color{violet}\normalfont\normalsize\bfseries}
{\color{violet}\thesection}{1em}{}
% Comment out the above 5 lines for black and white

\begin{document}

\raggedright
\footnotesize
\begin{multicols*}{3}

% multicol parameters
% These lengths are set only within the two main columns
%\setlength{\columnseprule}{0.25pt}
\setlength{\premulticols}{1pt}
\setlength{\postmulticols}{1pt}
\setlength{\multicolsep}{1pt}
\setlength{\columnsep}{1pt}

%%%%%%%%%%%%%%%%%%%%%%%%%%%%%%%%%%%%
%%% TITLE
%%%%%%%%%%%%%%%%%%%%%%%%%%%%%%%%%%%%

\begin{center}
    {\color{blue} \Large{\textbf{Mathematical Methods for Quantitative Finance}}} 
\end{center}

%%%%%%%%%%%%%%%%%%%%%%%%%%%%%%%%%%%%
%%% ATTRIBUTIONS
%%%%%%%%%%%%%%%%%%%%%%%%%%%%%%%%%%%%

\scriptsize

Cheat sheet for MITx 15.455x Mathematical Methods for Quantitative Finance.


%%%%%%%%%%%%%%%%%%%%%%%%%%%%%%%%%%%%
%%% BEGIN CHEATSHEET
%%%%%%%%%%%%%%%%%%%%%%%%%%%%%%%%%%%%
\section{Week 5: It\^{o} Calculus)}\smallskip \hrule height 1pt \smallskip

\subsection{Black-Scholes equation}

\begin{description}[topsep=0pt]
	\item[Summary of some key formulas] ~
	\begin{itemize}[topsep=0pt]
	 \item It\^{o} process: $\text{d}X = a \text{d}t+b \text{d}B$\
     \item It\^{o} formula: \begin{align*}
     	\text{d}F &= \frac{\partial F}{\partial t} \text{d}t + \frac{\partial^2 F}{\partial X^2} \text{d}X + \frac{b^2}{2} \frac{\partial F}{\partial X} \text{d}t \\
 	         	&=   \Bigg( \frac{\partial F}{\partial t} +a \frac{\partial F}{\partial X} + \frac{b^2}{2} \frac{\partial^2 F}{\partial X^2} \Bigg) \text{d}t + b \frac{\partial F}{\partial X} \text{d}B   
 \end{align*}
      
      \item Stock price: $\text{d}S = \mu S \text{d}t + \sigma \text{d}B \implies \text{d}(\text{log } S) = \Big(\mu - \frac{\sigma^2}{2} \Big) \text{d}t + \sigma \text{d}B$
      \item Black-Scholes:  $\Delta = \partial V / \partial S$ , $\text{d}\pi=r \pi \text{d}t$  ,
   
      $$ \frac{\partial V}{\partial t} + \frac{\sigma^2 S^2}{2} \frac{\partial ^2 V}{\partial S^2} + r S \frac{\partial V}{\partial S} -rV =0 $$ 
	\end{itemize}
\end{description}



\subsection{Recitation 5}

\begin{description}[topsep=0pt]
	\item[Expectations from Brownian integrals] ~
	\begin{itemize}[topsep=0pt]
		\item $ dB \sim N(0,dt) $
		\item $  \int_{0}^{t} dB = B_t - B_0 \sim N(0,t ) $
		\item $E[f(B_t-B_0)] = E[f(\sqrt{t} z)] = \frac{1}{\sqrt{2\pi}} \int e^{-z^2/2}f(\sqrt{t}z) dz$ 
		\item Example  $E[f(B_t-B_0)] = E[(B_t-B_0)^4] = E[(\sqrt{t} z)^4] = t^2 E[z^4] = 3t^2 $ \\
		
		We can pull out $\sqrt{t}$,which is nonstochastic to get $t^2 E[z^4]$
		, $E[z^4]$ is a well-known Gaussian integal that we use in the kurtosis=3.
		\item Useful formula:  $E[e^{\alpha z + \beta}]  = e^{\alpha^2/2 + \beta}$
	\end{itemize}
\end{description}

\section{Week 6: Continuous-Time Finance}\smallskip \hrule height 1pt \smallskip

\subsection{It\^{o} processes in higher dimensions}

\begin{description}[topsep=0pt]
	\item[It\^{o}'s lemma: multiple stochastic variables] ~
	\begin{itemize}[topsep=0pt]
		\item  $ \text{d}X_i = a_i(t,X_1,X_2,...) \text{d}t+ b_i(t,X_1,X_2,...)\text{d} B_i $ \\
		 $\text{d} F = \frac{\partial F}{\partial t} \text{d}t + \sum \frac{\partial F}{\partial X_i} \text{d}X_i + \frac{1}{2} \sum \rho_{ij} b_i b_j \frac{\partial ^2 F}{\partial X_i X_j} \text{d}t$ 
		 \item Heuristics "rule of thum" for correlated Brownian motions : \\
		    $ (\text{d}B_i)^2 \rightarrow \text{d}t $ , 
		    $ (\text{d}B_i)(\text{d}B_j) \rightarrow \rho_{ij} \text{d}t$ ,\\ 
		    $ (\text{d}X_i )^2 \rightarrow b^2_i \text{d}t $ ,
		    $ (\text{d}X_i)(\text{d}X_j) \rightarrow \rho_{ij} b_i b_j \text{d} t $ 
		 \item two stochastic variables case:  $\text{d}X_1 = a_1 \text{d}t + b_1 \text{d}B_1 $, $\text{d}X_2 = a_2 \text{d}t + b_2 \text{d}B_2 $ \\
		 $\text{d}F =  \frac{\partial F}{\partial t} \text{d}t + \frac{\partial F}{\partial X_1} \text{d}X_1+ \frac{\partial F}{\partial X_2} \text{d}X_2 + \frac{b_1^2}{2}  \frac{\partial^2 F}{\partial X_1^2} + \frac{b_2^2}{2}  \frac{\partial^2 F}{\partial X_2^2} + b_1 b_2 \rho \frac{\partial^2 F}{\partial X_1 \partial X_2}  $
		 
	\end{itemize}
\end{description}

\newpage

\section{Recommended Resources} \smallskip \hrule height 1pt \smallskip

\bigskip

\begin{itemize}
\item MITx 15.455x 
MITx 15.455x
Mathematical Methods for Quantitative Finance [Lecture Slides] (\url{https://learning.edx.org/course/course-v1:MITx+15.455x+3T2020/home})

\item Tsay, Analysis of Financial Time Series (3e), Wiley. (Tsay)
\item Capinski and Zastawniak, Mathematics for Finance, Springer. (CZ)
\item Olver, Introduction to Partial Differential Equations (2016), Springer. (Olver)
\item Campbell, Lo, and MacKinlay, Econometrics of Financial Markets (1997), Princeton. (CLM)
\item Lang, Introduction to Linear Algebra (2e), Springer (Lang)
\item Axler, Linear Algebra Done Right (3e), Springer (Axler)
\item LaTeX File (\texttt{\href{https://github.com/j053g/cheatsheets/blob/main/15.455x/15.455x_math_methods_for_quant_finance.tex}{github.com/j053g/cheatsheets/15.455x}})
\end{itemize}

\begin{center}
	\emph{Last Updated \today}
\end{center}

\end{multicols*}



\end{document}
