\section{Options, Part 2}

\subsection*{Binomial model: risk-neutral pricing}


Solve by replication $\delta$ shares of the stock, $b$ dollars of riskless bond

$\delta u S_0 + b (1+r) = C_u$ \\
$\delta d S_0 + b (1+r) = C_d$

Solution: $\delta = \frac{C_u-C_d}{(u-d)S_0}$ , 
$b=\frac{1}{1+r}\frac{uC_d-dC)u}{u-d}$ \\
Then: $C_0 = \delta S_0 + b$ \\
$ C_0 = \frac{C_u-C_d}{(u-d)} \frac{1}{1+r} \frac{uC_d-dC)u}{u-d}$ \\

\subsection*{Risk neutral probability}

$q_u=\frac{(1+r)-d}{u-d}$, $q_d = \frac{u-(1+r)}{u-d}$ Then

$C_0=\frac{q_u C_u + q_d C_d}{1+r} = \frac{E^Q[C_T]}{1+r}$ 
where $E^Q[\cdot]$ is the expectation under probability $Q=(1,1-q)$, which is
called the risk-neutral probability

\subsection*{State prices and risk-neutral probabilities}
$\Phi_u = \frac{q}{1+r}$ , $\Phi_d = \frac{1-q}{1+r}$

$\Phi_{uu} = \frac{q^2}{(1+r)^2}$ , $\Phi_{ud} = \Phi_{du} = \frac{q(1-q)}{(1+r)^2}$ , $\Phi_{dd} = \frac{(1-q)^2}{(1+r)^2}$
With state prices, can price any state-contingent payoff as a portfolio of
state-contingent claims: mathematically equivalent to the risk-neutral
valuation formula.


\subsection*{Implementing binominal model}

\begin{itemize}
	\item As we reduce the length of the time step, holding the maturity fixed, the
	binomial distribution of log returns converges to Normal distribution.
	\item Key model parameters $u$, and $d$ need to be chosen to reflect the
	distribution of the stock return
\end{itemize}

Once choice is:  $u=exp(\sigma \frac{T}{n})$, $d=\frac{1}{u}$, $p=\frac{1}{2} + \frac{1}{2} \frac{u}{\sigma} \sqrt{\frac{T}{n}}$

\subsection*{Black-Scholes-Merton formula}

$C_0 = S_0 N(x) - K e^{-rT} N (x-\sigma \sqrt(T))$ \\
$x = \frac{ln(\frac{S_0}{K e^{-rT}})}{\sigma \sqrt{T}} + \frac{1}{2} \sigma \sqrt{T}$ \\

In Excel \texttt{$N(x)$=NORM.S.DIST(x,TRUE)}
The call is equivalent to a levered long position in the stock;
 $S_0 N(x) $ is the amount invested in the stock;
$K e^{-rT} N (x-\sigma \sqrt(T))$ is the dollar amount borrowed.

Equivalent formulation:
$ C(S_t, t) = N(d_1)S_t - N(d_2)Ke^{-r(T - t)} $ \\
$ d_1 = \frac{1}{\sigma\sqrt{T - t}}\left[\ln\left(\frac{S_t}{K}\right) + \left(r + \frac{\sigma^2}{2}\right)(T - t)\right] $\\
$ d_2 = d_1 - \sigma\sqrt{T - t} $ 

The price of a corresponding put option based on put-call parity  is:
$ P(S_t, t) = Ke^{-r(T - t)} - S_t + C(S_t, t) =  N(-d_2) Ke^{-r(T - t)} - N(-d_1) S_t $

\subsection*{Option Greeks}

Delta: $\delta = \frac{\partial C}{\partial S}$
Omega: $\Omega = \frac{\partial C}{\partial S} \frac{S}{C}$
Gamma: $\Gamma = \frac{\partial \delta}{\partial S} = \frac{\partial^2 C}{\partial S^2} $
Theta: $\Theta = \frac{\partial C}{\partial S}$
Vega:  $\mathcal{V} = \frac{\partial C}{\partial S}$
