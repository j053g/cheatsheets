\section{Payout \& Risk Management}

\subsection*{Modigliani-Miller on payout}

MM Payout Policy Irrelevance: In a financial market with no imperfections,
holding fixed its investment policy (hence its free cash flow), a firm's payout
policy is irrelevant and does not affect its initial share price.

Paying dividends is a zero NPV transaction. Firm value before dividend = Firm value dividend + Dividend.


\subsection*{Hedging basics}

Let $V_{original}$: Value of the original position (unhedged), \\
$V_{hedging}$ - Value of the hedging position, \\
$V_{net}$ -  Value of the hedged position.
Then, \\
$V_{net} = V_{original} + (\text{hedge ratio}) \times V_{hedging} $

The hedge is perfect if: 1. $V_{original}$ and $V_{hedging}$ are perfectly correlated, and 2. Hedge ratio is appropriately chosen.
Otherwise, the hedging is imperfect.


\subsection*{Managing interest rate risk}


\begin{center}
	\begin{tabular}{ |c|c|c|c| } 
		\hline
		Bond & Price & Dur & ModDur \\ 
		\hline
		A & $B_A$ & $D_A$ &  $MD_A$ \\ 
		B & $B_B$ & $D_B$ &  $MD_B$ \\ 
		\hline
	\end{tabular}
\end{center}

$V_P=V_A+V_B = n_AB_A + n_B B_B$ \\
$MD_P = \frac{V_A}{V_A+V_B} MD_A + \frac{V_B}{V_A+V_B}MD_B$ \\

$\delta$ is the hedge ratio if bond A is used to hedge bond B  $MD_B-\delta MD_A=0$, then  $\delta = \frac{MD_B}{MD_A}$


\section{From FMF I}

\subsection*{Arbitrage Pricing}

Example for three assets: riskless bonds pays \$100 in each state currently traded at $B1_0$, stock 1 pays off $[S1_1, S1_2, S1_3]$ and currently traded at $S1_0$, 
stock 2 pays off $[S2_1, S2_2, S2_3]$ and currently traded at $S2_0$:
$$
\begin{pmatrix}
	100 & 100 & 100   \\
	S1_1 & S1_2 & S1_3 \\
	S2_1 & S2_2 & S3_3 \\							
\end{pmatrix} \cdot 
\begin{pmatrix}
	\phi_1 \\
	\phi_2 \\
	\phi_3 \\								
\end{pmatrix}	
=
\begin{pmatrix}
	B1_0\\
	S1_0 \\
	S2_0 \\								
\end{pmatrix}
$$ 
solve system to find state prices $\phi_1$, $\phi_2$, $\phi_3$.	

\subsection*{Interest Rate Risk measures}
{\bf Modified Duration (MD) for discount bond $ B_t=\frac{1}{(1+y)^t} $ } $ MD(B_t) = -\frac{1}{B_t}\frac{dB_t}{dy} = \frac{t}{1+y}$  \\
{\bf Macaulay Duration} is the weighted average term to maturity  $ D = \sum_{t=1}^{T} \left(   \frac{PV(CF_T)}{B}  t \right)  =   \frac{1}{B}\sum_{t=1}^{T} \left(  \frac{CF_t}{(1+y)^t} t \right)  $ \\
{\bf Modified Duration} measures bond's interest rate risk by its relative price change with respect
to a unit change in yield (with a negative sign):  $ MD =  -\frac{1}{B}\frac{dB}{dy}  = \frac{D}{1+y}  $ \\
{\bf Convexity (CX)} measure the curvature of the bond price as function of the yield:  $ CX =  \frac{1}{2}\frac{1}{B}\frac{d^2B}{dy^2}  $ \\
$ CX = \frac{1}{2} \frac{1}{P} \frac{1}{(1+y)^2} \sum_{t=1}^{T} \frac{t (t+1) CF_t}{(1+t)^t} =  \frac{1}{2} \frac{1}{P} \frac{1}{(1+y)^2} \sum_{t=1}^{T} PV(CF_t) t (t+1)  $ \\
{\bf Taylor series approximation of bond price changes}  $ \Delta B \approx  B \left(  -MD  \cdot \Delta y + CX \cdot ( \Delta y)^2 \right)    $ \\




\subsection*{Growth Opportunities and Stock Valuation}

\begin{itemize}
	\item {\bf P/E and PVGO:} $ P_0 = \frac{EPS_1}{r} + PVGO $
	\item {\bf if $PVGO=0$:} $ P/E = \frac{1}{r}  $
	\item {\bf if $PVGO>0$:} $ P/E = \frac{1}{r} + \frac{PVGO}{EPS_2} > \frac{1}{r} $
\end{itemize}
Investment and Growth
\begin{itemize}
	\item {plow-back ratio} $b_t = 1-payout = 1-\frac{DIV}{EPS}$
	\item {\bf Investments:} $ I_t = EPS_t \cdot b_t $
	\item {\bf Next year earnings} $ EPS_{t+1} = EPS_t +ROI_t \cdot I_t $
	\item {\bf Next year book value:} $ BVPS_{t+1} = BVPS_t + I_{t+1} $
	\item {\bf Dividends:} $ D_t = EPS_t (1-b_t) $
	\item {\bf Growh rate:} $g = b \cdot ROI$
	\item  $V_{0,no inv t>0}=E_1/r$ , $NPV_1 = -E_1 + \frac{E_2-E_1}{r}$ ,  $V_0 = V_{0,no inv t>0}+\frac{NPV_1}{1+r}$		
\end{itemize}


\subsection*{NPV Rule}
$NPV = CF_0 + \frac{CF_1}{1+r_1}+\frac{CF_2}{(1+r_2)^2} + \cdots + \frac{CF_T}{(1+r_T)^T} $

	$CF = (1-\tau) (OperatingProfits) - CapEx + \tau \cdot Depreciation -\Delta WC $\\
	$WC = Inventory + A/R - A/P$, $A/R : \text{Accounts Receivable}$, $A/P : \text{Accounts Payable}$
	
	