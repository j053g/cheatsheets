\section{Financing, Part II}

\subsection*{MM with taxes}

Notation: $U$: Unlevered, $L$: Levered, 
 $X$: Terminal value, $\tau$: taxes. \\
Firm $U$: $(1-\tau)X$ \\
Firm $L$: $(1-\tau)(X-r_D D) + r_D D = (1-\tau)X + \tau r_D D $ \\
$V_U =\frac{(1-\tau)X}{1+r_A} $ \\
$V_L =\frac{(1-\tau)X}{1+r_A} + \frac{\tau r_D D}{1+r_D} = V_U + \frac{\tau r_D D}{1+r_D}  $ \\


The value of a levered firm equals the value of the unlevered
firm (with the same assets) plus the present value of the {\bf tax shield}

$V_L = V_U + PV(\text{debt tax shield}) = V_U + PVTS$

\subsection*{Costs of financial distress }

$V_L = V_U + PV(\text{debt tax shield}) - PV (\text{cost of financial distress}) = APV$. Where $APV$: Adjusted present value.

\subsection*{MM with personal tax}

Notation: Debt level at $D$ with interest rate $r_D$.
Corporate tax rate $\tau$. Investors pay additional personal taxes:
\begin{itemize}
	\item Tax rate on equity (dividend and capital gain) $\pi$.
	\item Tax rate on debt (interest) $\delta$.
\end{itemize}

total after-tax cashflow:

$\underbrace{(1-\pi)(1-\tau)X}_{\text{all equity firm}} + \underbrace{[(1-\delta)-(1-\pi)(1-\tau)]r_DD}_{\text{tax impact of debt}}$

$V_L = V_U + [(1-\delta)-(1-\pi)(1-\tau)]PV(r_DD) $