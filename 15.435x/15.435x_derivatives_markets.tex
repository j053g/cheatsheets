\documentclass[10pt,landscape,a4paper]{article}
\usepackage[english]{babel}
\usepackage[utf8]{inputenc}
\usepackage[plain]{algorithm}
\usepackage[noend]{algpseudocode}
\usepackage{tikz}
\usepackage{pgfplots}
\usepackage{palatino}
\usepackage{multicol}
\usepackage{blkarray}

\usepackage{calc}
\usepackage{ifthen}
\usepackage[landscape]{geometry}
\usepackage{graphicx}
\usepackage{amsmath, amssymb, amsthm}
\DeclareMathOperator*{\argmin}{argmin}
\DeclareMathOperator*{\argmax}{argmax}

\usepackage{physics}
\usepackage{latexsym, marvosym}
\usepackage{pifont}
\usepackage{lscape}
\usepackage{dsfont}
\usepackage{graphicx}
\usepackage{array}
\usepackage{booktabs}
\usepackage[bottom]{footmisc}
\usepackage{tikz}
\usetikzlibrary{shapes}
\usepackage{pdfpages}
\usepackage{wrapfig}
\usepackage{enumitem}
\setlist[description]{leftmargin=0pt}
\usepackage{xfrac}
\usepackage[pdftex,
            pdfauthor={Jose},
            pdftitle={Derivatives Markets: Advanced Modeling and Strategies},
            pdfsubject={Cheat sheet for MITx 15.435x Derivatives Markets: Advanced Modeling and Strategies},
            pdfkeywords={finance} {cheatsheet} {pdf} {cheat} {sheet} {formulas} {equations}
            ]{hyperref}
\usepackage[
            open,
            openlevel=2
            ]{bookmark}
\usepackage{relsize}
\usepackage{rotating}

% for monospace font https://tex.stackexchange.com/questions/50810/good-monospace-font-for-code-in-latex
\usepackage{inconsolata}
\usepackage[gen]{eurosym}

 \newcommand\independent{\protect\mathpalette{\protect\independenT}{\perp}}
    \def\independenT#1#2{\mathrel{\setbox0\hbox{$#1#2$}%
    \copy0\kern-\wd0\mkern4mu\box0}} 
            
\newcommand{\noin}{\noindent}    
\newcommand{\logit}{\textrm{logit}} 
%\newcommand{\var}{\textrm{Var}}
\newcommand{\cov}{\textrm{Cov}} 
\newcommand{\corr}{\textrm{Corr}} 
\newcommand{\N}{\mathcal{N}}
\newcommand{\Bern}{\textrm{Bern}}
\newcommand{\Bin}{\textrm{Bin}}
\newcommand{\Beta}{\textrm{Beta}}
\newcommand{\Gam}{\textrm{Gamma}}
\newcommand{\Expo}{\textrm{Expo}}
\newcommand{\Pois}{\textrm{Pois}}
\newcommand{\Unif}{\textrm{Unif}}
\newcommand{\Geom}{\textrm{Geom}}
\newcommand{\NBin}{\textrm{NBin}}
\newcommand{\Hypergeometric}{\textrm{HGeom}}
\newcommand{\HGeom}{\textrm{HGeom}}
\newcommand{\Mult}{\textrm{Mult}}

\geometry{top=.4in,left=.2in,right=.2in,bottom=.4in}

\pagestyle{empty}
\makeatletter
\renewcommand{\section}{\@startsection{section}{1}{0mm}%
                                {-1ex plus -.5ex minus -.2ex}%
                                {0.5ex plus .2ex}%x
                                {\normalfont\large\bfseries}}
\renewcommand{\subsection}{\@startsection{subsection}{2}{0mm}%
                                {-1explus -.5ex minus -.2ex}%
                                {0.5ex plus .2ex}%
                                {\normalfont\normalsize\bfseries}}
\renewcommand{\subsubsection}{\@startsection{subsubsection}{3}{0mm}%
                                {-1ex plus -.5ex minus -.2ex}%
                                {1ex plus .2ex}%
                                {\normalfont\small\bfseries}}
\makeatother

\setcounter{secnumdepth}{0}

\setlength{\parindent}{0pt}
\setlength{\parskip}{0pt plus 0ex}

% -----------------------------------------------------------------------

\usepackage{titlesec}

\titleformat{\section}
{\color{blue}\normalfont\normalsize\bfseries}
{\color{blue}\thesection}{0em}{}
\titleformat{\subsection}
{\color{violet}\normalfont\normalsize\bfseries}
{\color{violet}\thesection}{1em}{}
% Comment out the above 5 lines for black and white

\begin{document}

\raggedright
\footnotesize
\begin{multicols*}{3}

% multicol parameters
% These lengths are set only within the two main columns
%\setlength{\columnseprule}{0.25pt}
\setlength{\premulticols}{1pt}
\setlength{\postmulticols}{1pt}
\setlength{\multicolsep}{1pt}
\setlength{\columnsep}{1pt}

%%%%%%%%%%%%%%%%%%%%%%%%%%%%%%%%%%%%
%%% TITLE
%%%%%%%%%%%%%%%%%%%%%%%%%%%%%%%%%%%%

\begin{center}
    {\color{blue} \Large{\textbf{Derivatives Markets: Advanced Modeling and Strategies}}} 
\end{center}

%%%%%%%%%%%%%%%%%%%%%%%%%%%%%%%%%%%%
%%% ATTRIBUTIONS
%%%%%%%%%%%%%%%%%%%%%%%%%%%%%%%%%%%%

\scriptsize

Cheat sheet for MITx 15.435x Derivatives Markets: Advanced Modeling and Strategies.


%%%%%%%%%%%%%%%%%%%%%%%%%%%%%%%%%%%%
%%% BEGIN CHEATSHEET
%%%%%%%%%%%%%%%%%%%%%%%%%%%%%%%%%%%%

\section{Week 1: Forward Contracts}\smallskip \hrule height 1pt \smallskip

\subsection{Forward contract basics}

\begin{description}[topsep=0pt]
	\item[Forward Contract] ~
	\begin{itemize}[topsep=0pt]
		\item A \textbf{forward contract} is an agreement between two counterparties to trade a pre-
		specified amount of goods or securities at a pre-specified future date, $T$, for a pre-
		specified price, $F_0$.
		\item The \textbf{Profit/Loss (P/L)} at the contract maturity T for each counterparty is:
		$P/L_{long} = N (S_T - F_0)$  , $P/L_{short} = N (F_0- S_T)$    
		\item  Price of a zero coupong bond with face value $Z$:  $P=e^{-r_T T}Z $\\
		$f(0,T_1,T_2)$ 	denotes the \textbf{forward rate} between time $T_1$ and $T_2$, as of
		time 0 : $f(0,T_1,T_2) = \frac{T_2 r_{T_2} - T_1 r_{T_1}}{T_2-T_1}$  
		\item Long forward positions are equivalent to borrowing and going long
		in the underlying asset
		\item Forward short positions are equivalent to lending and going short
		the underlying		
	\end{itemize}
\end{description}
 
 \subsection{Pricing formulas}
 
 \begin{description}[topsep=0pt]
 	\item[Pricing formulas] ~
 	\begin{itemize}[topsep=0pt]
 		\item An \textbf{arbitrage opportunity} is a trading strategy that either
 		(1) Yields a positive profit today, and zero cash flows in the future; or
 		(2) Costs nothing today and yields a positive profit in the future
 		\item \textbf{The Law of One Price}:
 			Securities with identical payoffs must have the same price
 		\item \textbf{Stock}  with known dividend $D$ at time $t < T$ :  $F_0 = (P_{S,0} - De^{-rt}) e^{rT}$ \\
 		Stock with known dividend yield $q$: $F_0 = P_{S,0}e^{(r-q)T}$
 		\item \textbf{Bond}  with coupon $C$ at time $t < T$: $F_0 = (P_{B,0} - Ce^{-rt}) e^{rT}$ 
 		\item \textbf{Currencies}.  $r_\$$ ($ r_{\euro} $)  the USD (EUR) risk-free rate.  $S_t$ is the exchange rate (USD per EUR) at time $t$:  $F_0 = S_0 e^{(r_\$ - r_{\euro}) T}$
 	\end{itemize}
 \end{description}
 
  \begin{description}[topsep=0pt]
 	\item[Forward prices for commodities] ~
 	\begin{itemize}[topsep=0pt]
 		\item Forward price with lump-sum storage cost $U$:  $F_{0,T} = (S_0 + PV(U)) e^{rT}$
 		\item Forward price with proportional storage cost $u$: $F_{0,T} = S_0 e^{(r+u)T}$
 		\item Forward price with convenience yield $y$: $F_{0,T} = S_0 e^{(r-y)T}$
 		\item Forward price with proportional storage cost and convenience yield: $F_{0,T} = S_0 e^{(r+u-y)T}$
 		\item \textbf{Contango}  is a pattern of forward prices that increases with contract maturity
 		\item \textbf{Backwardation} is a pattern of forward prices over time that decreases with
 		contract maturity
 	\end{itemize}
 \end{description}

\subsection{Key concepts for hedging and speculating}
  \begin{description}[topsep=0pt]
	\item[Valuing a forward contract over time] ~
	\begin{itemize}[topsep=0pt]
		\item Suppose that $K=F_0$ the original delivery price, initial value of contract $f_0=0$.
		\item Value of a \textbf{long} forward contract at time $t$ :  $f_{long,t,T} = (F_t-K) e ^{-r(T-t)}$ 
		\item Value of a \textbf{short} forward contract at time $t$ :  $f_{short,t,T} = (K-F_t) e ^{-r(T-t)}$ 
		\item \textbf{Basis} is the difference between the spot and forward price
		of a security or commodity.
		\item \textbf{Cross-hedging} involves using a contract type to hedge
		which differs from the security or commodity being hedged.
		\item The \textbf{hedge ratio} is the relative number of forward contracts to units
		of the asset being hedged that maximizes the effectiveness of the hedge:  $N_S \mathbb{E}[\text{d}S] = N_F \mathbb{E}[\text{d}F] $ then: $\frac{N_S}{N_F} = \frac{\mathbb{E}[\text{d}F]}{\mathbb{E}[\text{d}S]}$. If long in spot then short in forwards, and vice versa.
	\end{itemize}
\end{description}
 
\newpage

\section{Recommended Resources} \smallskip \hrule height 1pt \smallskip

\bigskip

\begin{itemize}
\item MITx 15.435x 
Derivatives Markets: Advanced Modeling and Strategies \href{https://learning.edx.org/course/course-v1:MITx+15.435x+1T2021/home}{Lecture Slides}

\item John Hull’s, Options Futures and Other Derivatives, 10th edition
\item Bruce Tuckman and Angel Serrat, Fixed Income Securities; Tools for Today’s Markets, 3rd Edition (BTAS) 
\item LaTeX File (\texttt{\href{https://github.com/j053g/cheatsheets/blob/main/15.435x/15.435x_derivatives_markets.tex}{github.com/j053g/cheatsheets/15.435x}})
\end{itemize}

\begin{center}
	\emph{Last Updated \today}
\end{center}

\end{multicols*}



\end{document}
