\documentclass[10pt,landscape,a4paper]{article}
\usepackage[english]{babel}
\usepackage[utf8]{inputenc}
\usepackage[plain]{algorithm}
\usepackage[noend]{algpseudocode}
\usepackage{tikz}
\usepackage{pgfplots}
\usepackage{palatino}
\usepackage{multicol}
\usepackage{blkarray}

\usepackage{calc}
\usepackage{ifthen}
\usepackage[landscape]{geometry}
\usepackage{graphicx}
\usepackage{amsmath, amssymb, amsthm}
\DeclareMathOperator*{\argmin}{argmin}
\DeclareMathOperator*{\argmax}{argmax}

\usepackage{physics}
\usepackage{latexsym, marvosym}
\usepackage{pifont}
\usepackage{lscape}
\usepackage{dsfont}
\usepackage{graphicx}
\usepackage{array}
\usepackage{booktabs}
\usepackage[bottom]{footmisc}
\usepackage{tikz}
\usetikzlibrary{shapes}
\usepackage{pdfpages}
\usepackage{wrapfig}
\usepackage{enumitem}
\setlist[description]{leftmargin=0pt}
\usepackage{xfrac}
\usepackage[pdftex,
            pdfauthor={Jose},
            pdftitle={Derivatives Markets: Advanced Modeling and Strategies},
            pdfsubject={Cheat sheet for MITx 15.435x Derivatives Markets: Advanced Modeling and Strategies},
            pdfkeywords={finance} {cheatsheet} {pdf} {cheat} {sheet} {formulas} {equations}
            ]{hyperref}
\usepackage[
            open,
            openlevel=2
            ]{bookmark}
\usepackage{relsize}
\usepackage{rotating}

% for monospace font https://tex.stackexchange.com/questions/50810/good-monospace-font-for-code-in-latex
\usepackage{inconsolata}
\usepackage[gen]{eurosym}

 \newcommand\independent{\protect\mathpalette{\protect\independenT}{\perp}}
    \def\independenT#1#2{\mathrel{\setbox0\hbox{$#1#2$}%
    \copy0\kern-\wd0\mkern4mu\box0}} 
            
\newcommand{\noin}{\noindent}    
\newcommand{\logit}{\textrm{logit}} 
%\newcommand{\var}{\textrm{Var}}
\newcommand{\cov}{\textrm{Cov}} 
\newcommand{\corr}{\textrm{Corr}} 
\newcommand{\N}{\mathcal{N}}
\newcommand{\Bern}{\textrm{Bern}}
\newcommand{\Bin}{\textrm{Bin}}
\newcommand{\Beta}{\textrm{Beta}}
\newcommand{\Gam}{\textrm{Gamma}}
\newcommand{\Expo}{\textrm{Expo}}
\newcommand{\Pois}{\textrm{Pois}}
\newcommand{\Unif}{\textrm{Unif}}
\newcommand{\Geom}{\textrm{Geom}}
\newcommand{\NBin}{\textrm{NBin}}
\newcommand{\Hypergeometric}{\textrm{HGeom}}
\newcommand{\HGeom}{\textrm{HGeom}}
\newcommand{\Mult}{\textrm{Mult}}

\geometry{top=.4in,left=.2in,right=.2in,bottom=.4in}

\pagestyle{empty}
\makeatletter
\renewcommand{\section}{\@startsection{section}{1}{0mm}%
                                {-1ex plus -.5ex minus -.2ex}%
                                {0.5ex plus .2ex}%x
                                {\normalfont\large\bfseries}}
\renewcommand{\subsection}{\@startsection{subsection}{2}{0mm}%
                                {-1explus -.5ex minus -.2ex}%
                                {0.5ex plus .2ex}%
                                {\normalfont\normalsize\bfseries}}
\renewcommand{\subsubsection}{\@startsection{subsubsection}{3}{0mm}%
                                {-1ex plus -.5ex minus -.2ex}%
                                {1ex plus .2ex}%
                                {\normalfont\small\bfseries}}
\makeatother

\setcounter{secnumdepth}{0}

\setlength{\parindent}{0pt}
\setlength{\parskip}{0pt plus 0ex}

% -----------------------------------------------------------------------

\usepackage{titlesec}

\titleformat{\section}
{\color{blue}\normalfont\normalsize\bfseries}
{\color{blue}\thesection}{0em}{}
\titleformat{\subsection}
{\color{violet}\normalfont\normalsize\bfseries}
{\color{violet}\thesection}{1em}{}
% Comment out the above 5 lines for black and white

\begin{document}

\raggedright
\footnotesize
\begin{multicols*}{3}

% multicol parameters
% These lengths are set only within the two main columns
%\setlength{\columnseprule}{0.25pt}
\setlength{\premulticols}{1pt}
\setlength{\postmulticols}{1pt}
\setlength{\multicolsep}{1pt}
\setlength{\columnsep}{1pt}

%%%%%%%%%%%%%%%%%%%%%%%%%%%%%%%%%%%%
%%% TITLE
%%%%%%%%%%%%%%%%%%%%%%%%%%%%%%%%%%%%

\begin{center}
    {\color{blue} \Large{\textbf{Derivatives Markets: Advanced Modeling and Strategies}}} 
\end{center}

%%%%%%%%%%%%%%%%%%%%%%%%%%%%%%%%%%%%
%%% ATTRIBUTIONS
%%%%%%%%%%%%%%%%%%%%%%%%%%%%%%%%%%%%

\scriptsize

Cheat sheet for MITx 15.435x Derivatives Markets: Advanced Modeling and Strategies.


%%%%%%%%%%%%%%%%%%%%%%%%%%%%%%%%%%%%
%%% BEGIN CHEATSHEET
%%%%%%%%%%%%%%%%%%%%%%%%%%%%%%%%%%%%






\section{Week 1: Forward Contracts}\smallskip \hrule height 1pt \smallskip

\subsection{Forward contract basics}

\begin{description}[topsep=0pt]
	\item[Forward Contract] ~
	\begin{itemize}[topsep=0pt]
		\item A \textbf{forward contract} is an agreement between two counterparties to trade a pre-
		specified amount of goods or securities at a pre-specified future date, $T$, for a pre-
		specified price, $F_0$.
		\item The \textbf{Profit/Loss (P/L)} at the contract maturity T for each counterparty is:
		$P/L_{long} = N (S_T - F_0)$  , $P/L_{short} = N (F_0- S_T)$    
		\item  Price of a zero coupong bond with face value $Z$:  $P=e^{-r_T T}Z $\\
		$f(0,T_1,T_2)$ 	denotes the \textbf{forward rate} between time $T_1$ and $T_2$, as of
		time 0 : $f(0,T_1,T_2) = \frac{T_2 r_{T_2} - T_1 r_{T_1}}{T_2-T_1}$  
		\item Long forward positions are equivalent to borrowing and going long
		in the underlying asset
		\item Forward short positions are equivalent to lending and going short
		the underlying		
	\end{itemize}
\end{description}
 
 \subsection{Pricing formulas}
 
 \begin{description}[topsep=0pt]
 	\item[Pricing formulas] ~
 	\begin{itemize}[topsep=0pt]
 		\item An \textbf{arbitrage opportunity} is a trading strategy that either
 		(1) Yields a positive profit today, and zero cash flows in the future; or
 		(2) Costs nothing today and yields a positive profit in the future
 		\item \textbf{The Law of One Price}:
 			Securities with identical payoffs must have the same price
 		\item \textbf{Stock}  with known dividend $D$ at time $t < T$ :  $F_0 = (P_{S,0} - De^{-rt}) e^{rT}$ \\
 		Stock with known dividend yield $q$: $F_0 = P_{S,0}e^{(r-q)T}$
 		\item \textbf{Bond}  with coupon $C$ at time $t < T$: $F_0 = (P_{B,0} - Ce^{-rt}) e^{rT}$ 
 		\item \textbf{Currencies}.  $r_\$$ ($ r_{\euro} $)  the USD (EUR) risk-free rate.  $S_t$ is the exchange rate (USD per EUR) at time $t$:  $F_0 = S_0 e^{(r_\$ - r_{\euro}) T}$
 	\end{itemize}
 \end{description}
 
  \begin{description}[topsep=0pt]
 	\item[Forward prices for commodities] ~
 	\begin{itemize}[topsep=0pt]
 		\item Forward price with lump-sum storage cost $U$:  $F_{0,T} = (S_0 + PV(U)) e^{rT}$
 		\item Forward price with proportional storage cost $u$: $F_{0,T} = S_0 e^{(r+u)T}$
 		\item Forward price with convenience yield $y$: $F_{0,T} = S_0 e^{(r-y)T}$
 		\item Forward price with proportional storage cost $u$ and convenience yield $y$: $F_{0,T} = S_0 e^{(r+u-y)T}$
 		\item \textbf{Contango}  is a pattern of forward prices that increases with contract maturity
 		\item \textbf{Backwardation} is a pattern of forward prices over time that decreases with
 		contract maturity
 	\end{itemize}
 \end{description}

\subsection{Key concepts for hedging and speculating}
  \begin{description}[topsep=0pt]
	\item[Valuing a forward contract over time] ~
	\begin{itemize}[topsep=0pt]
		\item Suppose that $K=F_0$ the original delivery price, initial value of contract $f_0=0$.
		\item Value of a \textbf{long} forward contract at time $t$ :  $f_{long,t,T} = (F_t-K) e ^{-r(T-t)}$ 
		\item Value of a \textbf{short} forward contract at time $t$ :  $f_{short,t,T} = (K-F_t) e ^{-r(T-t)}$ 
		\item \textbf{Basis} is the difference between the spot and forward price
		of a security or commodity.
		\item \textbf{Cross-hedging} involves using a contract type to hedge
		which differs from the security or commodity being hedged.
		\item The \textbf{hedge ratio} is the relative number of forward contracts to units
		of the asset being hedged that maximizes the effectiveness of the hedge:  $N_S \mathbb{E}[\text{d}S] = N_F \mathbb{E}[\text{d}F] $ then: $\frac{N_S}{N_F} = \frac{\mathbb{E}[\text{d}F]}{\mathbb{E}[\text{d}S]}$. If long in spot then short in forwards, and vice versa.
	\end{itemize}
\end{description}


\section{Week 2: Futures and Swaps Contracts}\smallskip \hrule height 1pt \smallskip

\subsection{Futures}

\begin{description}[topsep=0pt]
	\item[Forward Contract] ~
	\begin{itemize}[topsep=0pt]
		\item Daily settlement of gains and losses and have to maintain a minimum balance in a margin account.
		\item If margin balance falls below the maintenance margin, so the investors has
		to deposit into the account to restore the initial margin requirement.
	\end{itemize}
\end{description}

\subsection{Swaps}

\begin{description}[topsep=0pt]
	\item[Swap Pricing] ~
	\begin{itemize}[topsep=0pt]
		\item \textbf{Interest Rate Swap:} Given spot yield curve $s_1, s_2, \dots, s_N$ the coupon rate of the swap solves 
		$ F =\frac{cF}{(1+s_1)} + \frac{cF}{(1+s_2)^2} + \dots + \frac{(1+c)F}{(1+s_N)^N} $, solving for $c$: $c=\frac{1-B_N}{\sum_{i=1}^{N}B_i}$, where $B_i=\frac{1}{(1+s_i)^i}$ is the discount factor for period $i$.
		\item \textbf{Currency Swap}: The currency swap rate equals the current exchange rate multiplied by the
		ratio of the relative risk-free borrowing costs in the two currencies. Example: US firm pays bank  $1M \euro$ on $T = 0.5, 1, \dots , 2.5$. US Bank firm $1M \cdot K\$ $  then:   $K = S_0 \frac{e^{-0.5r_{\euro}}+e^{-1r_{\euro}}+\dots +e^{-2.5r_{\euro}}}{e^{-0.5r_{\$}}+e^{-1r_{\$}}+\dots +e^{-2.5r_{\$}}}$
		
	\end{itemize}
\end{description}


\section{Week 3 – Duration and convexity-based strategies for risk management}\smallskip \hrule height 1pt \smallskip

\subsection{Duration and Convexity}

\begin{description}[topsep=0pt]
	\item[Duration and Convexity] ~
	\begin{itemize}[topsep=0pt]
	\item {\bf Modified Duration (MD) for discount bond} $ P_t=\frac{1}{(1+y)^t} $, then $ MD(P_t) = -\frac{1}{P_t}\frac{dP_t}{dy} = \frac{t}{1+y}$
		\item {\bf Macaulay Duration} is the weighted average term to maturity  $ D = \sum_{t=1}^{T} \left(   \frac{PV(CF_T)}{P}  t \right)  =   \frac{1}{P}\sum_{t=1}^{T} \left(  \frac{CF_t}{(1+y)^t} t \right)  $
		\item {\bf Modified Duration} measures bond's interest rate risk by its relative price change with respect
		to a unit change in yield (with a negative sign):  $ D_M =  -\frac{1}{P}\frac{dP}{dy}  = \frac{D}{1+y}  $
		\item {\bf Convexity (CX)} measure the curvature of the bond price as function of the yield:  $ CX =  \frac{1}{2}\frac{1}{P}\frac{d^2B}{dy^2}  $ \\
		$$ CX = \frac{1}{2} \frac{1}{P} \frac{1}{(1+y)^2} \sum_{t=1}^{T} \frac{t (t+1) CF_t}{(1+t)^t} =  \frac{1}{2} \frac{1}{P} \frac{1}{(1+y)^2} \sum_{t=1}^{T} PV(CF_t) t (t+1)  $$
		\item {\bf Taylor series approximation of bond price changes}  $ \Delta P \approx  P \left(  -D_M  \cdot \Delta y + CX \cdot ( \Delta y)^2 \right)    $
		\item {\bf Dollar Duration:} Dollar duration is the modified duration multiplied by the price: $D_d = D_M \cdot P$. It's useful for hedging strategies and for understanding risk of zero NPV portolios.
	\end{itemize}
\end{description}

\subsection{Hedging}

\begin{description}[topsep=0pt]
	\item[Delta/Gamma Hedging] ~
	\begin{itemize}[topsep=0pt]
		\item A \textbf{delta neutral} portfolio equates the hedge ratio of assets and liabilities : $P_{A}D_{M,A} = P_{L}D_{M,L}$
		\item A \textbf{Gamma neutral } portfolio is delta neutral and equates gammas of assets and liabilities. Example hedge Liability $L$ with $D_{M,L}$ and $C_L$, with two assets $A_1, A_2$ with $D_{M,1}  C_1$ and  $D_{M,2}, C_2$ then : 
		\\ equate delta:  $L D_{M,L} = A_1 D_{M,1}+ A_2 D_{M,2}$ 
		\\ equate gamma: $L C_L = A_1 C_1 + A_2 C_2$, i.e. solve system to find $A_1$, $A_2$:	
		$
		\begin{pmatrix}
			D_{M,1} & D_{M,2}  \\
			C_1 & C_2 \\						
		\end{pmatrix} \cdot 
		\begin{pmatrix}
			A_1 \\
			A_2 \\							
		\end{pmatrix}	
		=
		\begin{pmatrix}
			L D_{M,L}\\
			L C_L							
		\end{pmatrix}
		$
		\item \textbf{Swap Dollar Duration} for fix receiver: $D_{dollar,rec}=P_{fix}D_{M,fix} - P_{flt}D_{eff,flt}$,  The effective duration of a (pure) floating rate bond is the time until the next reset, divided by $1 + \frac{Y_{APR}}{k}$,  $k$ is the number of compounding periods in a year, i.e.: $D_{eff,flt} = \frac{t_{nextreset}}{1 + \frac{Y_{APR}}{k}}$, for a new swap $P_{flt}=P{fix}=1$, then:
		\\$D_{dolar,rec}=D_{M,fix} - D_{eff,flt}$
	\end{itemize}
\end{description}


\section{Week 4: Options Strategies and Pricing Basics}\smallskip \hrule height 1pt \smallskip
 
 
 \begin{description}[topsep=0pt]
 	\item[Option Basics] ~
 	\begin{itemize}[topsep=0pt]
 		\item Put call parity: $Put-Call=e^{-rT}(K-F_{0,T})$ 
 		\item For a non-dividend paying stock: $Put = Call + e^{-rT}K-S_0$
 		\item \textit{Important: This formula only holds for European options!}
 	\end{itemize}
 \end{description}
 
  
 \begin{description}[topsep=0pt]
 	\item[Option Strategies] ~
 	\begin{itemize}[topsep=0pt]
 		\item Protective put:  Long put, Long stock. Payoff at  $T$: $ S_T + \operatorname{max}(K-S_T , 0) $
 		\item Covered call: Long stock, Short call.  Payoff at $T$: $ S_T – \operatorname{max}(S_T – K, 0) $
 		\item Bear spread: Short OTM put (strike $K_1$) and long ITM put ( $K_2 > K_1$)
 		\item Bull spreads: Long ITM call (strike $K_1$) and short OTM call ($ K_2 > K_1$)
 		\item Buttefly spread: Long 1 call with strike $K_0$, short 2 calls with strike $K_1$ and long 1 call with strike $K_2$, with $K_0<K_1<K_2$ and $K_1=\frac{K_0+K_2}{2}$
 		\item Straddle: Bet on high volatility. Long a call and a put with the same strike.
 		\item Strangle: Bet on high movements. Long put with $K_0$ and call with $K_1>K_0$ 
 		
 		
 	\end{itemize}
 \end{description}
 
 
  \begin{description}[topsep=0pt]
 	\item[Binomial trees] ~
 	\begin{itemize}[topsep=0pt]
 		\item One step: $S_0=\frac{E[S_1]}{1+R} = \frac{q S_{1,u}+(1-q)S_{1,d}}{1+R}$
 		\item Expected (gross) Return: $\mathbb{E}\Big[\frac{S_1}{S_0}\Big]=q \frac{S_{1,u}}{S_0}+(1-q)\frac{S_{1,d}}{S_0}$
 		\item Variance: $\mathbb{E}\Big[ \Big(\frac{S_1}{S_0} - \mathbb{E}\big[\frac{S_1}{S_0}\big] \Big)^2\Big]=q \Big(\frac{S_{1,u}}{S_0} -\mathbb{E}\big[\frac{S_1}{S_0}\big]\Big)^2+(1-q) \Big(\frac{S_{1,d}}{S_0}-\mathbb{E}\big[\frac{S_1}{S_0}\big]\Big)^2$
 		\item \textbf{replicating portfolio} :\\
 		$\Delta \cdot  S_{1,u} + B_0 e^{rT} = V_{1,u}$ \\
 		$\Delta \cdot  S_{1,d} + B_0 e^{rT} = V_{1,d}$
 		
 		Solution: $\Delta = \frac{ V_{1,u}- V_{1,d}}{S_{1,u}-S_{1,d}}$ , then we solve for $B_0=e^{-rT}(V_{1,u}-\Delta \cdot  S_{1,u})$
 		no arbitrage $\implies V_0= \Delta  \cdot S_0 + B_0$
 		\item \textbf{risk neutral pricing}: we choose $q^*$ so that all risky assets earn the risk-free rate:
 		$q^*S_{1,u} e^{-rT} + (1-q*)S_{1,d}e^{-rT}=S_0 \implies q^* = \frac{S_0 e^{rT}-S_{1,d}}{S_{1,u}-S_{1,d}}$ \\
 		$S_0 = \mathbb{E}^*[ e^{-rT} S_1]$. In general: $\textbf{Price of derivative} = \mathbb{E}^*[e^{-rT} \textbf{payoff}]$
 		\item American options.  Compare the value of immediate exercise with the value of the option. Exercise if an only if (for put): \textbf{ $K-S>$Discounted value of future distribution of payoffs if wait.}
 		\item \textbf{multi-step trees}:  $(i,j)$ time: $i=0,1,2,\dots, n$ ; node: $j=1,2,...n$ \\
 		with European derivative: $V^E_{i,j} = e^{-rh} \mathbb{E}^*[V^E_{i+1}|(i,j)]$, where $h=\frac{T}{n}$ \\
 		with American derivative: $V^A_{i,j} = \operatorname{max} \Big(g_{i,j}, e^{-rh} \mathbb{E}^*[V^A_{i+1}|(i,j)]\Big)$, where $h=\frac{T}{n}$ where $g_{i,j}$ is the payoff from the American derivative\\
 		
 		
 	\end{itemize}
 \end{description}
 
\newpage

\section{Recommended Resources} \smallskip \hrule height 1pt \smallskip

\bigskip

\begin{itemize}
\item MITx 15.435x 
Derivatives Markets: Advanced Modeling and Strategies \href{https://learning.edx.org/course/course-v1:MITx+15.435x+1T2021/home}{Lecture Slides}

\item John Hull’s, Options Futures and Other Derivatives, 10th edition
\item Bruce Tuckman and Angel Serrat, Fixed Income Securities; Tools for Today’s Markets, 3rd Edition (BTAS) 
\item LaTeX File (\texttt{\href{https://github.com/j053g/cheatsheets/blob/main/15.435x/15.435x_derivatives_markets.tex}{github.com/j053g/cheatsheets/15.435x}})
\end{itemize}



\begin{center}
	\emph{Last Updated \today}
\end{center}

\end{multicols*}



\end{document}
